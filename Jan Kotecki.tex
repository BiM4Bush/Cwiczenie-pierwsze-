\documentclass{article}
\usepackage{polski}
\begin{document}
\title{Krótka ściąga do LateXa}

\author{Jan Kotecki}
\date{\today}

\maketitle
\section{Przydatne rzeczy}
\section{Więcej przydatnych rzeczy}
\section{Jeszcze więcej przydatnych rzeczy}

\begin{center}
\newpage

\title{\Large Najczęściej używane komendy}
\maketitle
\end{center}

\begin{center}
\large
\begin{tabular}{|r|l|}
\hline
Komenda & Funkcja komendy \\
\hline
\textbackslash documentclass & Określa jaki dokument chcemy zrobić\\
\textbackslash usepackage & dodaje paczki np. z polskimi znakami\\
sdfsdgsdfgsgsg\\
\textbackslash end{document} & zakończenie dokumentu \\
\textbackslash author & dodaje autora dokumentu \\
\textbackslash title & dodaje tytuł dokumentu \\
\textbackslash section & tworzy spis \\
\textbackslash subsection & tworzy podpunkt w spise \\
\textbackslash maketitle & tworzy tytuł \\
\textbackslash newline & przeniesienie do nowej linijki \\
\textbackslash newpage & przeniesienie do nowej strony \\
\textbackslash thispagestyle & zmiana stylu dla aktualnej strony \\
\textbackslash date & dodaje date \\
\textbackslash begin(tabular)(r l) & tworzy tabele z 2 kolumanmi \\
\textbackslash end{tabular} & zamyka rozpoczętą tabele \\
\textbackslash hline & tworzy poziomą linie \\
\textbackslash Naprawde? & chce sie panu to czytać? \\
\textbackslash label & tworzy etykietę \\
\hline



\end{tabular}
\end{center}
\begin{center}
\title{\Huge Najważniejsze Litery Greckie}
\maketitle
\end{center}


\large
\begin{center}
\begin{tabular}{|r|l|}

\hline
litera & zapis w LateXie\\
\hline
$\alpha$ & \textbackslash alpha \\
$\beta$ & \textbackslash beta \\
$\gamma$ & \textbackslash gamma \\
$\delta$ & \textbackslash delta \\
$\epsilon$ & \textbackslash epsilon \\
$\varepsilon$ & \textbackslash varepsilon \\
$\zeta$ & \textbackslash zeta \\
$\eta$ & \textbackslash eta \\
\Huge $\kappa$ & \Huge \textbackslash kappa \\
$\pi$ & \textbackslash pi \\
$\lambda$ & \textbackslash lambda \\
$\varphi$ & \textbackslash varphi\\
$\omega$ & \textbackslash omega\\
\hline
\end{tabular}
\end{center}
\newpage
\begin{center}
\title{\Large ZNAKI MATEMATYCZNE}
\end{center}
\begin{center}
\begin{tabular}{|r|l|}
\hline
Znak & Zapis w LateXie \\
\hline
$\leq$ & \textbackslash leq \\
$\geq$ & \textbackslash geq \\
$\not\leq$ & \textbackslash not\textbackslash geq \\
$\equiv$ & \textbackslash equiv \\
$\not\equiv$ & \textbackslash not\textbackslash equiv \\
$\sim$ & \textbackslash sim \\
$\not\sim$ & \textbackslash not\textbackslash sim\\
$\approx$ & \textbackslash approx \\
$\not\approx$ & \textbackslash not\textbackslash approx \\
$\subset$ & \textbackslash subset \\
$\subseteq$ & \textbackslash subseteq \\
$\in$ & \textbackslash in \\
$\ni$ & \textbackslash ni \\
$\perp$ & \textbackslash perp \\
$\Omega$ & \textbackslash Omega \\
$\Pi$ & \textbackslash Pi \\
$\Delta$ & \textbackslash Delta \\
$\Lambda$ & \textbackslash Lambda \\
$\Sigma$ & \textbackslash Sigma \\
$\Phi$ & \textbackslash Phi \\
$\dashv$ & \textbackslash dashv \\
$\bigcirc$ & \textbackslash bigcirc \\
$\ast$ & \textbackslash ast \\
$\times$ & \textbackslash times\\ 
$\cdot$ & \textbackslash cdot\\
$\leftarrow$ & \textbackslash leftarrow \\
$\rightarrow$ & \textbackslash rightarrow \\
$\dagger$ & \textbackslash dagger \\
$\clubsuit$ & \textbackslash clubsuit\\
$\diamondsuit$ & \textbackslash diamondsuit \\
$\spadesuit$ & \textbackslash spadesuit\\
$\heartsuit$ & \textbackslash heartsuit \\
$\copyright$ & \textbackslash copyright \\

\hline

\end{tabular}
\end{center}

\newpage
\begin{center}
\title{\textbf{Jak formatować tekst?} }
\end{center}

\begin{center}
\title{\huge O tak :)}
\maketitle
\end{center}
\begin{center}
\begin{tabular}{|r|l|}
\hline
Funkcja & komenda \\
\hline
wyrównanie do lewego marginesu & \textbackslash flushleft \\
wyrównanie do prawego marginesu & \textbackslash flushright \\
Wyśrodkowanie tekstu & \textbackslash begin(center) "tekst" end(center)\\
zmienienie czcionki & w zależności od rozmiaru wpisujemy \\
\textbackslash tiny & najmniejsza \\
\textbackslash small & mała \\
\textbackslash normalsize & normalny \\
\textbackslash large & duża \\
\textbackslash Large & większa \\
\textbackslash LARGE & SERIO duża \\
\textbackslash huge & DUŻA duża\\
\textbackslash Huge & najwieksza\\
Zmiana czcionki & Komenda inna dla kazdej \\
\textbackslash textrm(tekst) & pismo proste \\
\textbackslash textsl(tekst) & \textsl{Pismo pochylone}\\
\textbackslash textbf & \textbf{pismo pogrubione} \\
\textbackslash textit & \textit{kursywa}\\
\hline


\end{tabular}
\end{center}
\begin{center}
\textbf{Mam nadzieję,że mieszczę się w wyznaczonej wielkości pracy,
Dziekuję za sprawdzenie.}
\end{center}


\end{document}
